%%
%% Author: thompson
%% 07.12.17
%%

% Preamble
\documentclass[11pt]{article}

% Packages
\usepackage{a4wide}
\usepackage[utf8]{inputenc}
\usepackage[ngerman]{babel}

\usepackage{amsmath}
\usepackage{scrextend}
\usepackage{enumitem}
\usepackage{verbatim}
\usepackage{standalone}

% Document
\begin{document}
    \section{Übungsblatt 08}
    \subsection{Das UDP-Protokoll}
    Wenden Sie Ihr Wissen aus der Vorlesung an und sehen Sie sich zusätzlich RFC 768 (User Datagram
    Protocol - UDP) an und beantworten Sie folgende Fragen:
    \begin{enumerate}[label=\textbf{\alph*}.)]
        \item \textbf{Welche Funktionalität stellt UDP (nicht) zur Verfügung?}\\
        Das User Datagram Protokol, kurz UDP, soll Datentransfer innerhalb eines Netzwerkes ermöglichen, mit minimalen
        Protokollmechanismen. Es baut auf dem Internet Protokol(IP) auf.
        Es ist eine verbindungslose Kommunikation, was bedeutet dass der Empfänger seine
        Empfangsbereitschaft, etwa via 'Handshake' wie bei TCP, nicht signalisiert haben muss, bevor das Packet gesendet wird.
        Dadurch gibt es auch keinen Verbindungsstatus, weder am Sender noch am Empfänger.
        UDP ermöglicht auch keine Congestion oder Flow Control, weshalb Daten unwiederruflich verloren gehen können.
        IP spezifiziert UDP-Packete als Protokol Nr. 17 (21 Octal).


        \item \textbf{Wie sehen die UDP-Header aus?}\\
        Das UDP-Format\footnote[1]{https://tools.ietf.org/html/rfc768} sieht wie folgt aus:
        \begin{verbatim}
            0       7 8     15 16    23 24     31
            +--------+--------+--------+--------+
            |     Source      |   Destination   |
            |      Port       |      Port       |
            +--------+--------+--------+--------+
            |                 |                 |
            |     Length      |    Checksum     |
            +--------+--------+--------+--------+
            |
            |          Application Data
            +---------------- ...
        \end{verbatim}
        \begin{addmargin}[1em]{1em}
            - Source Port: Optional - Default-Value = 0\\
            - Destination Port: Sets Receiver\\
            - Length: Length of the Datagram, including Header \& Data measured in Octets. Minimum = 8 Octetcs\\
            - Checksum: 16-Bit-Value to validate packet. Bei UDP Optional.
        \end{addmargin}

        \textbf{Berechnung einer Checksum}\\
        Die UDP-Checksum errechnet sich aus der Summe vom Komplement von 1, dem Pseudo-UDP Header(siehe unten) und dem UDP-Datagram.\\
        $\diamond$ Source Address: 32 Bits/4 Byte\\
        $\diamond$ Destination Address: 32 Bits/4 Byte\\
        $\diamond$ 'Zero' aka Reserved: 8 Bits/1 Byte\\
        $\diamond$ Protokol: 8 Bits/1 Byte\\
        $\diamond$ UDP Length: Header+Payload Length of actual Datagram.\\

        Sollte man die Checksum nicht setzen so wird diese '0', indem man die Checksum der einfachheitshalber
        auf 0xFFFF setzt.


        \underline{Example: Calculation from Source}
        ... Man definiert IP-Addressen:
        Source Address: 152.1.51.27
        Destination Address: 152.14.94.75
        ... Die Addressen geteilt in je 16-Bit:
        0x9801, 0x331b sowie 0x980e, 0x5e4b.
        ... Diese 4 Summiert ergibt:
        0x9801 + 0x331b + 0x980e + 0x5e4b = 0x1c175
        ... Beachte dass es einen Overflow auf 16 Bit gibt, dieser bleibt zunächst.
        ... Nun addiert man die UDP-Protokollnummer, also '17'
        17 = 0x0011 + 0x1c175 = 0x1c186
        ... Zuletzt addiert man die Länge des Datagrams. Für 10 Byte wäre dies:
        10 = 0x000a + 0x1c186 = 0x1c190
        % FIXME & TODO


        \item \textbf{Was ist ein Pseudo-Header?}\\
        Der Pseudoheader ist ein konzeptuelles Prefix zum UDP-header und beinhaltet Source Address,
        Destination Address, Protokoll und UDP-Länge. Dies bietet Schutz gegen fehlgeleitete Datagramme.
        Der Pseudo-UDP Header\footnote[1]{https://tools.ietf.org/html/rfc768} besteht aus 5 Feldern:
        \begin{verbatim}
            0       7 8     15 16    23 24    31
            +--------+--------+--------+--------+
            |          source address           |
            +--------+--------+--------+--------+
            |        destination address        |
            +--------+--------+--------+--------+
            |  zero  |protocol|   UDP length    |
            +--------+--------+--------+--------+
        \end{verbatim}

        \item \textbf{Wie groß können UDP-Segmente sinnvollerweise sein?}
        Die maximale UDP-Segmentsize ist vorgegeben durch das IP. Dementsprechend ist die Größe eines
        jeden UDP-Packetes zwischen 8 Byte und $2^(16)$ Byte.

        The correct maximum UDP message size is 65507, as determined by the following formula:
        0xffff - (sizeof(IP Header) + sizeof(UDP Header)) = 65535-(20+8) = 65507

        The maximum safe UDP payload is 508 bytes.
        This is a packet size of 576, minus the maximum 60-byte IP header and the 8-byte UDP header.
        Except on an IPv6-only route, where the maximum payload is 1212 bytes.

    \end{enumerate}

    Geben sie Beispiele an, wo UDP verwendet wird (wurde).
    UDP wird grundsätzlich in Verbindungen verwendet, wo eine schnelle Datenübertragung nötig ist.
    \begin{enumerate}[$\bullet$]
        \item Für simple Datenübertragungen kleinerer Einheiten.
        \item Bei schwacher Hardware
        \item Für sich schnell ändernde Daten: Wetterdaten, Spiele
        \item Mutlicast-Traffic: Daten werden an mehrere spezifizierte Klienten gesendet
        \item
    \end{enumerate}
    .. und ähnliche Instanzen.
    Grund für das Nutzen von UDP ist das Vermeiden des Overheads durch den TCP-Handshake und der marginale
    Packetverlust gilt als akzeptabel in vielen Applikationen.

    Erweiterungen des UDP, etwa Reliable User Datagram Protocol:RUDP oder UDP-Lite sind Beispiele für
    sicherere oder noch schnellere Variationen des UDP.

\end{document}
