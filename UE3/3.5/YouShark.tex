%%
%% Author: thompson
%% 03.11.17
%%

% Preamble
\documentclass[11pt]{article}

% Packages
\usepackage{a4wide}

% Document
\begin{document}
    \section{Analyse mit Wireshark}
    Nun werden wir heutigen Datenverkehr aufnehmen und analysieren. Via Youtube catchen wir
    einige Datenpackete welche wir anschließend begutachten, gemäß folgender Fragen:

    \begin{enumerate}
        \item Versuchen Sie die Verbindung über unterschiedliche Zugangsnetzwerke (z.B. LAN, WLAN, 3/4G sofern möglich) herzustellen und dokumentieren Sie allfällige Unterschiede.

        \item Welche Objekte werden vom Client via HTTP angefordert? Hinweis: nur jene beim Videostreaming, andere Objekte (z.B. HTML, Text, Bilder) können vernachlässigt werden.
            Wir setzen Wireshark auf unser Netzwerk an und starten  den Stream.
            Wir beenden die Aufnahme alsbald und analysieren unser .pcapng mithilfe von
        \begin{verbatim}
            - Analyze > Follow > TCP Stream / UDP Stream
            - Statistics > Conversations
            - ...
        \end{verbatim}
        % // TODO for another day. Odd internetconnectability right now.

    \end{enumerate}
\end{document}